% \iffalse
\let\negmedspace\undefined
\let\negthickspace\undefined
\documentclass[journal,12pt,twocolumn]{IEEEtran}
\usepackage{cite}
\usepackage{amsmath,amssymb,amsfonts,amsthm}
\usepackage{algorithmic}
\usepackage{graphicx}
\usepackage{textcomp}
\usepackage{xcolor}
\usepackage{txfonts}
\usepackage{listings}
\usepackage{enumitem}
\usepackage{mathtools}
\usepackage{gensymb}
\usepackage{comment}
\usepackage[breaklinks=true]{hyperref}
\usepackage{tkz-euclide} 
\usepackage{listings}
\usepackage{gvv}                                        
\def\inputGnumericTable{}                                 
\usepackage[latin1]{inputenc}                                
\usepackage{color}                                            
\usepackage{array}                                            
\usepackage{longtable}                                       
\usepackage{calc}                                             
\usepackage{multirow}                                         
\usepackage{hhline}                                           
\usepackage{ifthen}                                           
\usepackage{lscape}
\newtheorem{theorem}{Theorem}[section]
\newtheorem{problem}{Problem}
\newtheorem{proposition}{Proposition}[section]
\newtheorem{lemma}{Lemma}[section]
\newtheorem{corollary}[theorem]{Corollary}
\newtheorem{example}{Example}[section]
\newtheorem{definition}[problem]{Definition}
\newcommand{\BEQA}{\begin{eqnarray}}
\newcommand{\EEQA}{\end{eqnarray}}
\newcommand{\define}{\stackrel{\triangle}{=}}
\theoremstyle{remark}

\newtheorem{rem}{Remark}
\begin{document}
\parindent 0px
\bibliographystyle{IEEEtran}
\title{Assignment 11.9.5\_6Q}
\author{EE23BTECH11028 - Kamale Goutham$^{}$% <-this % stops a space
}
\maketitle
\newpage
\bigskip
\section*{Question}
A spiral is made up of successive semicircles,with centres alternately at A and B,starting with centre at A,of radii $0.5cm,1.0cm,1.5cm,2.0cm$,... as shown in Fig.$5.4$.what is the total length of such a spiral made up of thirteen consecutive semicircles?(Take $\pi=\frac{22}{7}$)\\
\begin{tikzpicture}

% Define the centers A and B
\coordinate (A) at (0,0);
\coordinate (B) at (0.5,0);

% Draw the spiral using semicircles
\foreach \r in {0.5,1.5} {
  \draw (A) ++(\r,0) arc (0:180:\r);
}
\foreach \r in {1.0,2.0} {
  \draw (B) ++(\r,0) arc (0:-180:\r);
}
\node[label={above:A}] at (-0.1,-0.6){};
\node[label={above:B}] at (0.5,-0.6){};
\node[label={above:l}] at (0.5,-1.7){};
\node[label={above:L}] at (0.2,0.3){};
% Draw the axes (optional)
\draw[->] (-3,0) -- (3,0) node[right] {$x$};
\draw[->] (0,-3) -- (0,3) node[above] {$y$};

\end{tikzpicture}
\end{figure}
\end{document}
