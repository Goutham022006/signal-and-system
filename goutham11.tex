% \iffalse
\let\negmedspace\undefined
\let\negthickspace\undefined
\documentclass[journal,12pt,twocolumn]{IEEEtran}
\usepackage{cite}
\usepackage{amsmath,amssymb,amsfonts,amsthm}
\usepackage{algorithmic}
\usepackage{graphicx}
\usepackage{textcomp}
\usepackage{xcolor}
\usepackage{txfonts}
\usepackage{listings}
\usepackage{enumitem}
\usepackage{mathtools}
\usepackage{gensymb}
\usepackage{comment}
\usepackage[breaklinks=true]{hyperref}
\usepackage{tkz-euclide} 
\usepackage{listings}
\usepackage{gvv}                                        
\def\inputGnumericTable{}                                 
\usepackage[latin1]{inputenc}                                
\usepackage{color}                                            
\usepackage{array}                                            
\usepackage{longtable}                                       
\usepackage{calc}                                             
\usepackage{multirow}                                         
\usepackage{hhline}                                           
\usepackage{ifthen}                                           
\usepackage{lscape}
\newtheorem{theorem}{Theorem}[section]
\newtheorem{problem}{Problem}
\newtheorem{proposition}{Proposition}[section]
\newtheorem{lemma}{Lemma}[section]
\newtheorem{corollary}[theorem]{Corollary}
\newtheorem{example}{Example}[section]
\newtheorem{definition}[problem]{Definition}
\newcommand{\BEQA}{\begin{eqnarray}}
\newcommand{\EEQA}{\end{eqnarray}}
\newcommand{\define}{\stackrel{\triangle}{=}}
\theoremstyle{remark}

\newtheorem{rem}{Remark}
\begin{document}
\parindent 0px
\bibliographystyle{IEEEtran}
\title{Assignment 11.9.5\_6Q}
\author{EE23BTECH11028 - Kamale Goutham$^{}$% <-this % stops a space
}
\maketitle
\newpage
\bigskip
\section*{Question}
Find the sum of all two digit numbers which when divided by 4,yields 1 as reminder?\\
\solution 
  \begin{align}
      \text{x(n)}=x(0)+n\times{\text{d}}
  \end{align}
  \begin{align}
    \text{n} = \frac{\text{x(n)} - \text{x(0)}}{\text{d}} 
  \end{align}
  \begin{align}
      n=\frac{97-13}{4}=21
  \end{align}
Input parameters are:\\

\begin{table}[ht]
    \centering
    \def\arraystretch{1.5}
    \footnotesize
\begin{tabular}{|p{2cm}|p{2.5cm}|p{2.3cm}|}
    \hline
    PARAMETER & VALUE & DESCRIPTION  \\ \hline
    $$x\brak0$$ & $$13$$ & First term \\ \hline
    $$d$$ & $$4$$ & common difference \\ \hline
    $$x(n)$$ & $$[13+4n]u\brak n$$ & General term of the series  \\ 
    \hline
  \end{tabular}
    \caption{Input Parameter TABLE}
    \label{tab:11.9.5.6}
\end{table}

\begin{align}
     x(z)&=\sum\limits^{\infty}_{k=-\infty}{x(k)\times u(k)\times(z^{-k}})
\end{align}\\
R.O.C$\rightarrow$$\mod{z}$ $\geq$ 1:\\
\begin{align}
x(z)&=13\times (z)(z-1)^{-1}+4\times(z)(z-1)^{-2}\\
x(z)&=\frac{13-9z^{-1}}{(1-z^{-1})^2},|z|>1
\end{align}
\begin{align}
    y(n)=&x(n)*u(n)\\\implies Y(Z)=&X(Z)U(Z)\\Y(Z)=&\brak{\frac{13-9z^{-1}}{(1-z^{-1})^2}} \brak{\frac{1}{1-z^{-1}}}\\Y(Z)=&\brak{\frac{13-9z^{-1}}{(1-z^{-1})^{3}}},|z|>1
\end{align}
Using contour integration to find the inverse z-transform,
\begin{align}
    y(21)=&\frac{1}{2\pi j}\oint_{C}Y(z) z^{20} dz  \\=&\frac{1}{2\pi j}\oint_{C}\brak{\frac{(13-9z^{-1})z^{20}}{(1-z^{-1})^{3}}}
\end{align}
We can observe that the pole is repeated $3$ times and thus $m=3$,
\begin{align}
    R&=\frac{1}{\brak {m-1}!}\lim\limits_{z\to a}\frac{d^{m-1}}{dz^{m-1}}\brak {{(z-a)}^{m}f\brak z}  \\&=\frac{1}{\brak {2}!}\lim\limits_{z\to 1}\frac{d^{2}}{dz^{2}}\brak {{(z-1)}^{3}\frac{(13-9z^{-1})z^{23}}{(z-1)^{3}}}\\&=\frac{1}{\brak {2}!}\lim\limits_{z\to 1}\frac{d^{2}}{dz^{2}}\brak{13z^{23}-9z^{22}}\\&=\frac{1}{\brak {2}!}\brak{13\times 23\times 22-9\times 22\times21}\\&=1210
\end{align}
Therefore, the sum of all two-digit numbers that, when divided by 4, yield a remainder of 1 is 1210.\\
\end{document}