% \iffalse
\let\negmedspace\undefined
\let\negthickspace\undefined
\documentclass[journal,12pt,twocolumn]{IEEEtran}
\usepackage{cite}
\usepackage{amsmath,amssymb,amsfonts,amsthm}
\usepackage{algorithmic}
\usepackage{graphicx}
\usepackage{textcomp}
\usepackage{xcolor}
\usepackage{txfonts}
\usepackage{listings}
\usepackage{enumitem}
\usepackage{mathtools}
\usepackage{gensymb}
\usepackage{comment}
\usepackage[breaklinks=true]{hyperref}
\usepackage{tkz-euclide} 
\usepackage{listings}
\usepackage{gvv}                                        
\def\inputGnumericTable{}                                 
\usepackage[latin1]{inputenc}                                
\usepackage{color}                                            
\usepackage{array}                                            
\usepackage{longtable}                                       
\usepackage{calc}                                             
\usepackage{multirow}                                         
\usepackage{hhline}                                           
\usepackage{ifthen}                                           
\usepackage{lscape}
\newtheorem{theorem}{Theorem}[section]
\newtheorem{problem}{Problem}
\newtheorem{proposition}{Proposition}[section]
\newtheorem{lemma}{Lemma}[section]
\newtheorem{corollary}[theorem]{Corollary}
\newtheorem{example}{Example}[section]
\newtheorem{definition}[problem]{Definition}
\newcommand{\BEQA}{\begin{eqnarray}}
\newcommand{\EEQA}{\end{eqnarray}}
\newcommand{\define}{\stackrel{\triangle}{=}}
\theoremstyle{remark}

\newtheorem{rem}{Remark}
\begin{document}
\parindent 0px
\bibliographystyle{IEEEtran}
\title{Assignment CS\_15Q}
\author{EE23BTECH11028 - Kamale Goutham$^{}$% <-this % stops a space
}
\maketitle
\newpage
\bigskip
\section*{Question}
The Lucas sequence $L_{n}$is defined by the recurrence relation:\\
\begin{align*}
    L_{n}=L_{n-1}+L_{n-2}, for n\geq3
\end{align*}
with $L_{1}$=1 and $L_{2}$=3\\
Which one of the option given is TRUE?\\
\begin{enumerate}
    \item $L_{n}=\brak{\frac{1+\sqrt{5}}{2}}^n+\brak{\frac{1-\sqrt{5}}{2}}^n$
    \item $L_{n}=\brak{\frac{1+\sqrt{5}}{2}}^n-\brak{\frac{1-\sqrt{5}}{3}}^n$
    \item $L_{n}=\brak{\frac{1+\sqrt{5}}{2}}^n+\brak{\frac{1-\sqrt{5}}{3}}^n$
    \item $L_{n}=\brak{\frac{1+\sqrt{5}}{2}}^n-\brak{\frac{1-\sqrt{5}}{2}}^n$
\end{enumerate}
\end{document}
