% \iffalse
\let\negmedspace\undefined
\let\negthickspace\undefined
\documentclass[journal,12pt,twocolumn]{IEEEtran}
\usepackage{cite}
\usepackage{amsmath,amssymb,amsfonts,amsthm}
\usepackage{algorithmic}
\usepackage{graphicx}
\usepackage{textcomp}
\usepackage{xcolor}
\usepackage{txfonts}
\usepackage{listings}
\usepackage{enumitem}
\usepackage{mathtools}
\usepackage{gensymb}
\usepackage{comment}
\usepackage[breaklinks=true]{hyperref}
\usepackage{tkz-euclide} 
\usepackage{listings}
\usepackage{gvv}                                        
\def\inputGnumericTable{}                                 
\usepackage[latin1]{inputenc}                                
\usepackage{color}                                            
\usepackage{array}                                            
\usepackage{longtable}                                       
\usepackage{calc}                                             
\usepackage{multirow}                                         
\usepackage{hhline}                                           
\usepackage{ifthen}                                           
\usepackage{lscape}
\newtheorem{theorem}{Theorem}[section]
\newtheorem{problem}{Problem}
\newtheorem{proposition}{Proposition}[section]
\newtheorem{lemma}{Lemma}[section]
\newtheorem{corollary}[theorem]{Corollary}
\newtheorem{example}{Example}[section]
\newtheorem{definition}[problem]{Definition}
\newcommand{\BEQA}{\begin{eqnarray}}
\newcommand{\EEQA}{\end{eqnarray}}
\newcommand{\define}{\stackrel{\triangle}{=}}
\theoremstyle{remark}

\newtheorem{rem}{Remark}
\begin{document}
\parindent 0px
\bibliographystyle{IEEEtran}
\title{Assignment IN\_40Q}
\author{EE23BTECH11028 - Kamale Goutham$^{}$% <-this % stops a space
}
\maketitle
\newpage
\bigskip
\section*{Question}
The signal $x(t)=(t-1)^2u(t-1)$,where u(t) is unit-step function,has the Laplace transform X(s).The Value of X(1) is 
\begin{enumerate}
    \item $\frac{1}{e}$
    \item $\frac{2}{e}$
    \item $2e$
    \item $e^2$
\end{enumerate}
\hfill{(GATE 2022 IN 40)}\\
\solution

\begin{align}
    x(t)&=(t-1)^2u(t-1) 
\end{align}
  Taking Laplace-Transform:\\
  
     $\mathcal{L}\{u(t)\}$
\begin{align}
    u(t) \system{L} \frac{1}{s}  
\end{align}
    $\mathcal{L}\{tu(t)\}$ 
\begin{align}
    tu(t) \system{L} \frac{1}{s^2}
\end{align}
\begin{center}
     $ \vdots \notag$ \\
\end{center}
  $\mathcal{L}\{t^nu(t)\}$ 
\begin{align}
    t^nu(t) \system{L} \frac{n!}{s^{n+1}} \label{IN_40.4}
\end{align}
if X(s) is Laplace transform of x(t) then,\\
\begin{align}
  x(t-t_0)&=e^{-st_0}X(s)\label{IN_40.5}
\end{align}
using \ref{IN_40.4} and \ref{IN_40.5}\\
 $\mathcal{L}\{tu(t)\}$ 
\begin{align}
    (t-1)^2u(t-1) \system{L} \frac{2e^{-s}}{s^3}
\end{align}
\begin{align}
    X(s)&=\frac{2e^{-s}}{s^3}\\
    X(1)&=\frac{2}{e}
\end{align}
$\therefore$ 2 is Correct.
\end{document}
